% This must be in the first 5 lines to tell arXiv to use pdfLaTeX, which is strongly recommended.
\pdfoutput=1
% In particular, the hyperref package requires pdfLaTeX in order to break URLs across lines.

\documentclass[11pt]{article}

% Remove the "review" option to generate the final version.
\usepackage[review]{ACL2023}

% Standard package includes
\usepackage{times}
\usepackage{latexsym}

% For proper rendering and hyphenation of words containing Latin characters (including in bib files)
\usepackage[T1]{fontenc}
% For Vietnamese characters
% \usepackage[T5]{fontenc}
% See https://www.latex-project.org/help/documentation/encguide.pdf for other character sets

% This assumes your files are encoded as UTF8
\usepackage[utf8]{inputenc}

% This is not strictly necessary, and may be commented out.
% However, it will improve the layout of the manuscript,
% and will typically save some space.
\usepackage{microtype}

% This is also not strictly necessary, and may be commented out.
% However, it will improve the aesthetics of text in
% the typewriter font.
\usepackage{inconsolata}


% If the title and author information does not fit in the area allocated, uncomment the following
%
%\setlength\titlebox{<dim>}
%
% and set <dim> to something 5cm or larger.

\title{Instructions for ACL 2023 Proceedings}

% Author information can be set in various styles:
% For several authors from the same institution:
% \author{Author 1 \and ... \and Author n \\
%         Address line \\ ... \\ Address line}
% if the names do not fit well on one line use
%         Author 1 \\ {\bf Author 2} \\ ... \\ {\bf Author n} \\
% For authors from different institutions:
% \author{Author 1 \\ Address line \\  ... \\ Address line
%         \And  ... \And
%         Author n \\ Address line \\ ... \\ Address line}
% To start a seperate ``row'' of authors use \AND, as in
% \author{Author 1 \\ Address line \\  ... \\ Address line
%         \AND
%         Author 2 \\ Address line \\ ... \\ Address line \And
%         Author 3 \\ Address line \\ ... \\ Address line}

\author{First Author \\
  Affiliation / Address line 1 \\
  Affiliation / Address line 2 \\
  Affiliation / Address line 3 \\
  \texttt{email@domain} \\\And
  Second Author \\
  Affiliation / Address line 1 \\
  Affiliation / Address line 2 \\
  Affiliation / Address line 3 \\
  \texttt{email@domain} \\}

\begin{document}
\maketitle
\begin{abstract}
This document is a supplement to the general instructions for *ACL authors. It contains instructions for using the \LaTeX{} style file for ACL 2023.
The document itself conforms to its own specifications, and is, therefore, an example of what your manuscript should look like.
These instructions should be used both for papers submitted for review and for final versions of accepted papers.
\end{abstract}

\section{Introduction}

\noindent \textbf{Proposal 1} --- spaCy performance on named entity recognition task with code-mixed data \bigbreak

The goal of the research is to evaluate how well spaCy performs on NER tasks when it comes to multilingual data, above all where two languages are used interchangeably in one sentence.  The English-Spanish code-mixed data used here comes directly from CALCS 2018 (Computational Approaches to Linguistic Code-Switching) shared task \citep{aguilaretal2018calcs}. Since spaCy out of the box is language specific, the transformer variants of the both involved languages will be used to tag the data. The results (two versions) will be compared to the gold labels given in the CoNLL-U file to retrieve accuracy and possibly other evaluation metrics.

\textbf{Possible expansion for AP} One interesting type of code-switching is the so-called ``insertional code-switching'', where a single token or a short phrase of the embedded language L2 appears inside a sentence structure of the matrix language L1. One could imagine that at least in some cases a language model trained only on L1 would falsely identify inserted L2 elements as special names. My question is, whether there is a correlation between the length of the inserted words and the possibility of them being regarded as named entities. For this the results from above will be re-analysed. The cases of L2 insertions which are not named entities themselves will be extracted with the corresponding word lengths and the annotations returned by spaCy (namely NE == True/False). The percentage of falsely annotated insertions on each word length can then be calculated. Using a correlation test, whether there is a correlation between inserted L2 word length and error rate can then be answered easily.
\vspace{20pt}

\noindent \textbf{Proposal 2} --- NLP model performance on sentiment analysis tasks with code-mixed data \bigbreak

The dataset collected for this proposal is Dataset for Sentiment Analysis on Code-Mix Telugu-English Text \citep{kusampudi2021cssentiment}. The goals and procedures of this BN base research will be largely similar to the previous one except for two critical points: 1) Instead of NER results, the sentiment pipeline provided by \texttt{SpacyTextBlob} will be called up. The results are encoded as \texttt{polarity}; 2) spaCy does not support Telugu. So only the English model can be used. Fundamentally this is a text classification task but with tricky data. Two aspects will be evaluated: 1) How well spaCy English model performs on sentences with English as L1, Telugu as L2; 2) How well the same model performs on sentences with Telugu as L1, English as L2. The results from the latter condition are expected to be much worse than the first one. Because the model itself is not trained on the matrix language, it should not be surprising if the sentiment information of the ``foreign language'' can not be captured effectively.

\textbf{Possible expansion for AP} The unpredictability of the classification results using spaCy lies in that spaCy models are trained solely on monolingual data. A possible way to run a ``real'' monolingual analysis would be integrating multilingual word vectors in the classifier training process. The retrieving of this kind of vectors is doable using language-agnostic models like those from \citet{smith2017multilang}, \citet{devlin2018mbert} and more recently \citet{conneau2019roberta}. My design for this project is to compare the performances of each embedding model on the list on sentiment analysis to determine how well they can capture the semantics of multilingual sentences. The procedure goes as follows: First the data will be split into train and test set. The embeddings of all sentences in both sets will be retrieved using different models. Train different classification models with train embeddings using scikit-learn. Get prediction accuracies on test set embeddings. The results will be organized in tabular form to provide an overview. It will be of great interest to see whether there is a consistent increase of accuracy with embeddings generated by newer models. spaCy's performance could also be brought into comparison as the baseline. One aspect worth further considering is whether a subset of spaCy predictions should be used as the baseline. Since its monolingual nature, including predictions on data with Telugu as matrix language will certainly bring unfair disadvantages to its results. One approach would be by reducing the data to English as L1 only and let spaCy English model compete against other models trained on multilingual embeddings of the same subset of sentences.



%
% These instructions are for authors submitting papers to ACL 2023 using \LaTeX. They are not self-contained. All authors must follow the general instructions for *ACL proceedings,\footnote{\url{http://acl-org.github.io/ACLPUB/formatting.html}} as well as guidelines set forth in the ACL 2023 call for papers.\footnote{\url{https://2023.aclweb.org/calls/main_conference/}} This document contains additional instructions for the \LaTeX{} style files.
% The templates include the \LaTeX{} source of this document (\texttt{acl2023.tex}),
% the \LaTeX{} style file used to format it (\texttt{acl2023.sty}),
% an ACL bibliography style (\texttt{acl\_natbib.bst}),
% an example bibliography (\texttt{custom.bib}),
% and the bibliography for the ACL Anthology (\texttt{anthology.bib}).
%
% \section{Engines}
%
% To produce a PDF file, pdf\LaTeX{} is strongly recommended (over original \LaTeX{} plus dvips+ps2pdf or dvipdf). Xe\LaTeX{} also produces PDF files, and is especially suitable for text in non-Latin scripts.
% \begin{table}
% \centering
% \begin{tabular}{lc}
% \hline
% \textbf{Command} & \textbf{Output}\\
% \hline
% \verb|{\"a}| & {\"a} \\
% \verb|{\^e}| & {\^e} \\
% \verb|{\`i}| & {\`i} \\
% \verb|{\.I}| & {\.I} \\
% \verb|{\o}| & {\o} \\
% \verb|{\'u}| & {\'u}  \\
% \verb|{\aa}| & {\aa}  \\\hline
% \end{tabular}
% \begin{tabular}{lc}
% \hline
% \textbf{Command} & \textbf{Output}\\
% \hline
% \verb|{\c c}| & {\c c} \\
% \verb|{\u g}| & {\u g} \\
% \verb|{\l}| & {\l} \\
% \verb|{\~n}| & {\~n} \\
% \verb|{\H o}| & {\H o} \\
% \verb|{\v r}| & {\v r} \\
% \verb|{\ss}| & {\ss} \\
% \hline
% \end{tabular}
% \caption{Example commands for accented characters, to be used in, \emph{e.g.}, Bib\TeX{} entries.}
% \label{tab:accents}
% \end{table}
% \section{Preamble}
% \begin{table*}
% \centering
% \begin{tabular}{lll}
% \hline
% \textbf{Output} & \textbf{natbib command} & \textbf{Old ACL-style command}\\
% \hline
% \citep{ct1965} & \verb|\citep| & \verb|\cite| \\
% \citealp{ct1965} & \verb|\citealp| & no equivalent \\
% \citet{ct1965} & \verb|\citet| & \verb|\newcite| \\
% \citeyearpar{ct1965} & \verb|\citeyearpar| & \verb|\shortcite| \\
% \citeposs{ct1965} & \verb|\citeposs| & no equivalent \\
% \citep[FFT;][]{ct1965} &  \verb|\citep[FFT;][]| & no equivalent\\
% \hline
% \end{tabular}
% \caption{\label{citation-guide}
% Citation commands supported by the style file.
% The style is based on the natbib package and supports all natbib citation commands.
% It also supports commands defined in previous ACL style files for compatibility.
% }
% \end{table*}
% The first line of the file must be
% \begin{quote}
% \begin{verbatim}
% \documentclass[11pt]{article}
% \end{verbatim}
% \end{quote}
% To load the style file in the review version:
% \begin{quote}
% \begin{verbatim}
% \usepackage[review]{ACL2023}
% \end{verbatim}
% \end{quote}
% For the final version, omit the \verb|review| option:
% \begin{quote}
% \begin{verbatim}
% \usepackage{ACL2023}
% \end{verbatim}
% \end{quote}
% To use Times Roman, put the following in the preamble:
% \begin{quote}
% \begin{verbatim}
% \usepackage{times}
% \end{verbatim}
% \end{quote}
% (Alternatives like txfonts or newtx are also acceptable.)
% Please see the \LaTeX{} source of this document for comments on other packages that may be useful.
% Set the title and author using \verb|\title| and \verb|\author|. Within the author list, format multiple authors using \verb|\and| and \verb|\And| and \verb|\AND|; please see the \LaTeX{} source for examples.
% By default, the box containing the title and author names is set to the minimum of 5 cm. If you need more space, include the following in the preamble:
% \begin{quote}
% \begin{verbatim}
% \setlength\titlebox{<dim>}
% \end{verbatim}
% \end{quote}
% where \verb|<dim>| is replaced with a length. Do not set this length smaller than 5 cm.
%
% \section{Document Body}
%
% \subsection{Footnotes}
%
% Footnotes are inserted with the \verb|\footnote| command.\footnote{This is a footnote.}
%
% \subsection{Tables and figures}
%
% See Table~\ref{tab:accents} for an example of a table and its caption.
% \textbf{Do not override the default caption sizes.}
%
% \subsection{Hyperlinks}
%
% Users of older versions of \LaTeX{} may encounter the following error during compilation:
% \begin{quote}
% \tt\verb|\pdfendlink| ended up in different nesting level than \verb|\pdfstartlink|.
% \end{quote}
% This happens when pdf\LaTeX{} is used and a citation splits across a page boundary. The best way to fix this is to upgrade \LaTeX{} to 2018-12-01 or later.
%
% \subsection{Citations}
%
%
%
% Table~\ref{citation-guide} shows the syntax supported by the style files.
% We encourage you to use the natbib styles.
% You can use the command \verb|\citet| (cite in text) to get ``author (year)'' citations, like this citation to a paper by \citet{Gusfield:97}.
% You can use the command \verb|\citep| (cite in parentheses) to get ``(author, year)'' citations \citep{Gusfield:97}.
% You can use the command \verb|\citealp| (alternative cite without parentheses) to get ``author, year'' citations, which is useful for using citations within parentheses (e.g. \citealp{Gusfield:97}).
%
% \subsection{References}
%
% \nocite{Ando2005,augenstein-etal-2016-stance,andrew2007scalable,rasooli-tetrault-2015,goodman-etal-2016-noise,harper-2014-learning}
%
% The \LaTeX{} and Bib\TeX{} style files provided roughly follow the American Psychological Association format.
% If your own bib file is named \texttt{custom.bib}, then placing the following before any appendices in your \LaTeX{} file will generate the references section for you:
% \begin{quote}
% \begin{verbatim}
% \bibliographystyle{acl_natbib}
% \bibliography{custom}
% \end{verbatim}
% \end{quote}
% You can obtain the complete ACL Anthology as a Bib\TeX{} file from \url{https://aclweb.org/anthology/anthology.bib.gz}.
% To include both the Anthology and your own .bib file, use the following instead of the above.
% \begin{quote}
% \begin{verbatim}
% \bibliographystyle{acl_natbib}
% \bibliography{anthology,custom}
% \end{verbatim}
% \end{quote}
% Please see Section~\ref{sec:bibtex} for information on preparing Bib\TeX{} files.
%
% \subsection{Appendices}
%
% Use \verb|\appendix| before any appendix section to switch the section numbering over to letters. See Appendix~\ref{sec:appendix} for an example.
%
% \section{Bib\TeX{} Files}
% \label{sec:bibtex}
%
% Unicode cannot be used in Bib\TeX{} entries, and some ways of typing special characters can disrupt Bib\TeX's alphabetization. The recommended way of typing special characters is shown in Table~\ref{tab:accents}.
%
% Please ensure that Bib\TeX{} records contain DOIs or URLs when possible, and for all the ACL materials that you reference.
% Use the \verb|doi| field for DOIs and the \verb|url| field for URLs.
% If a Bib\TeX{} entry has a URL or DOI field, the paper title in the references section will appear as a hyperlink to the paper, using the hyperref \LaTeX{} package.
%
% \section*{Limitations}
% ACL 2023 requires all submissions to have a section titled ``Limitations'', for discussing the limitations of the paper as a complement to the discussion of strengths in the main text. This section should occur after the conclusion, but before the references. It will not count towards the page limit.
% The discussion of limitations is mandatory. Papers without a limitation section will be desk-rejected without review.
%
% While we are open to different types of limitations, just mentioning that a set of results have been shown for English only probably does not reflect what we expect.
% Mentioning that the method works mostly for languages with limited morphology, like English, is a much better alternative.
% In addition, limitations such as low scalability to long text, the requirement of large GPU resources, or other things that inspire crucial further investigation are welcome.

% \section*{Ethics Statement}
% Scientific work published at ACL 2023 must comply with the ACL Ethics Policy.\footnote{\url{https://www.aclweb.org/portal/content/acl-code-ethics}} We encourage all authors to include an explicit ethics statement on the broader impact of the work, or other ethical considerations after the conclusion but before the references. The ethics statement will not count toward the page limit (8 pages for long, 4 pages for short papers).
%
% \section*{Acknowledgements}
% This document has been adapted by Jordan Boyd-Graber, Naoaki Okazaki, Anna Rogers from the style files used for earlier ACL, EMNLP and NAACL proceedings, including those for
% EACL 2023 by Isabelle Augenstein and Andreas Vlachos,
% EMNLP 2022 by Yue Zhang, Ryan Cotterell and Lea Frermann,
% ACL 2020 by Steven Bethard, Ryan Cotterell and Rui Yan,
% ACL 2019 by Douwe Kiela and Ivan Vuli\'{c},
% NAACL 2019 by Stephanie Lukin and Alla Roskovskaya,
% ACL 2018 by Shay Cohen, Kevin Gimpel, and Wei Lu,
% NAACL 2018 by Margaret Mitchell and Stephanie Lukin,
% Bib\TeX{} suggestions for (NA)ACL 2017/2018 from Jason Eisner,
% ACL 2017 by Dan Gildea and Min-Yen Kan, NAACL 2017 by Margaret Mitchell,
% ACL 2012 by Maggie Li and Michael White,
% ACL 2010 by Jing-Shin Chang and Philipp Koehn,
% ACL 2008 by Johanna D. Moore, Simone Teufel, James Allan, and Sadaoki Furui,
% ACL 2005 by Hwee Tou Ng and Kemal Oflazer,
% ACL 2002 by Eugene Charniak and Dekang Lin,
% and earlier ACL and EACL formats written by several people, including
% John Chen, Henry S. Thompson and Donald Walker.
% Additional elements were taken from the formatting instructions of the \emph{International Joint Conference on Artificial Intelligence} and the \emph{Conference on Computer Vision and Pattern Recognition}.

\newpage
% Entries for the entire Anthology, followed by custom entries
\bibliography{references}
\bibliographystyle{acl_natbib}

% \appendix
%
% \section{Example Appendix}
% \label{sec:appendix}

\end{document}
